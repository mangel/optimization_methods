\documentclass{article}

\usepackage{amsmath}
\usepackage{amsfonts}
\usepackage{amsthm}
\usepackage{amssymb}
\usepackage[margin=0.5in]{geometry}
\usepackage[spanish, mexico]{babel}
\usepackage[utf8]{inputenc}

\title{Taller}
\author{Miguel A. Gomez B.}

\begin{document}
	\maketitle
\paragraph{1} Demostrar una de las dos contentencias:
$$CC(A) \subseteq CO(A) \text{ o },$$
$$CO(A) \subseteq CC(A)$$
\begin{proof} Dado que el conjunto $A$ es convexo, existe una pareja de $x_n \in A$, tal que $x_a\lambda + (1-\lambda)x_b \in A$, con $\lambda \in [0,1]$ por la definición de conjunto convexo. Suponga ahora la siguiente combinación lineal:
	$$x_1\lambda_1 + x_2\lambda_2,$$
tal que la suma de los $\lambda_n = 1$, luego $\lambda_2 = 1 - \lambda_1$, entonces lo anterior se convierte en
$$x_1\lambda_1 + (1-\lambda_1) x_2$$
y que es la definición de combinación convexa, ahora supóngase bajo las mismas restricciones sobre $\lambda$, la combinación convexa:
$$x_1\lambda_1 + x_2\lambda_2 + \dots + x_n\lambda_n$$
reecribiremos $\lambda_2$ en términos de los demás lambdas:
\begin{align*}
	\lambda_2 &= 1 - \lambda_3 - \dots - \lambda_n - \lambda_1\\
	&= (1 - \lambda_3 - \dots - \lambda_n) - \lambda_1\\
	&= a - \lambda_1,
\end{align*}
de modo que ahora hemos construído un punto que por definición será convexo y por ende,
$$ x_1\lambda_1 + (a - \lambda_1)x_2 = x_a \in A,$$
de manera análoga se puede repetir el proceso anterior:
$$x_a(a-\lambda_3) + x_3 \lambda_3 + \dots + \lambda_n,$$
$$x_b(b-\lambda_4) + x_4 \lambda_4 + \dots + \lambda_n,$$
$$\dots,$$
hasta finalmente obtener la combinación:
$$x_z(z-\lambda_n) + x_n \lambda_n$$
y por ende toda combinación convexa estará contenida en el conjunto convexo.
\end{proof}
\paragraph{2} Encontrar las direcciones, direcciones extremas y puntos extremos de los conjuntos:
\begin{itemize}
	\item El conjunto factible dado en clase.
	\item $\{(x_1, x_2) \in \mathbb{R}: -x_1 \leq x_2\}$
\end{itemize}
\end{document}